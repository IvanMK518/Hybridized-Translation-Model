\section{Score Figures}
\label{sec:appendix}

\subsection{Natural Language Inference}

\begin{figure}[H]
    \centering
    \includegraphics[width=1\linewidth]{Screenshot 2024-11-25 at 1.36.10 PM.png}
    \caption{Average NLI F1 Score}
    \label{fig:nli-scores}
\end{figure}

\subsection{Machine Reading Comprehension}

\begin{figure}[H]
    \centering
    \includegraphics[width=1\linewidth]{Screenshot 2024-11-25 at 1.35.51 PM.png}
    \caption{Average MRC F1 Score}
    \label{fig:mrc-scores}
\end{figure}

\subsection{Machine Translation}

\begin{figure}[H]
    \centering
    \includegraphics[width=1\linewidth]{Screenshot 2024-11-25 at 1.35.37 PM.png}
    \caption{Average MT BLEU Score}
    \label{fig:mt-scores}
\end{figure}

\section{Prompts}

\subsection{Example Base Model Prompt}\\

\begin{enumerate}
    \item Observe the linguistic features of each entry of \{\textbf{D2}\}.
    \item Translate the entry into Haitian Creole and perform a linguistic analysis on the translation.
    \item Provide the cultural and conversational context of the translation.
    \item Provide a dialogue response in Haitian Creole.
\end{enumerate}\\

\subsection{Example LDP Model Prompt}\\

\begin{enumerate}
    \item Observe the linguistic features in \{\textbf{D2}\}.
    \item Translate the content of each \{\textbf{D2}\} entry into the language or dialect used in its corresponding \{\textbf{D1}\} entry.
    \item Provide a dialogue response or continuation that reflects the content and context of the \{\textbf{D1}\} entry while retaining the tone of \{\textbf{D2}\}.
    \item Ensure consistency with \{\textbf{D1}\}'s language and dialect style across all responses.
    \item Perform a cultural and conversational context analysis of \{\textbf{D1}\} based on \{\textbf{D2}\}.
    \item Provide a dialogue response in language \{\textbf{D1}\} (Haitian Creole) or a continuation based on what the speaker said in language \{\textbf{D2}\} (English).
\end{enumerate}\\

\subsection{Example PolyP Model Prompt}\\

\begin{enumerate}
    \item Conduct a translation task where you translate the dialect in \{\textbf{D1}\} to the dialect/language in \{\textbf{D3}\}.
    \item Perform a cultural and conversational context analysis of \{\textbf{D1}\} based on the shared tags with \{\textbf{D3}\} and report them.
    \item Provide a dialogue response in language \{\textbf{D1}\} or a continuation based on what the speaker said in \{\textbf{D3}\}.
    \item Re-prompt yourself with the same tasks using the language spoken in \{\textbf{D1}\}.
    \item Provide a dialogue response in language \{\textbf{D1}\} (Haitian Creole) or a continuation based on what the speaker said in language \{\textbf{D3}\} (AAVE).
\end{enumerate}\\

\section{Error Analysis}

\subsection{Natural Language Inference}

\begin{figure}[h]
    \centering
    \includegraphics[width=1\linewidth]{Screenshot 2024-12-12 at 6.22.36 PM.png}
    \caption{NLI Bin Analysis}
    \label{fig:enter-label}
\end{figure}

\subsection{Machine Reading Comprehension}

\begin{figure}[h]
    \centering
    \includegraphics[width=1\linewidth]{Screenshot 2024-12-12 at 6.21.56 PM.png}
    \caption{MRC Bin Analysis}
    \label{fig:enter-label}
\end{figure}

\subsection{Machine Translation}

\begin{figure}[h]
    \centering
    \includegraphics[width=1\linewidth]{Screenshot 2024-12-12 at 6.21.33 PM.png}
    \caption{MT Bin Analysis}
    \label{fig:enter-label}
\end{figure}

\clearpage

\subsection{Response Table}

\begin{table}[h] 
\centering
\renewcommand{\arraystretch}{1.4} 
\setlength{\tabcolsep}{5pt} 
\resizebox{\textwidth}{!}{%
\begin{tabular}{|c|p{4cm}|p{4.5cm}|p{4.5cm}|p{3.5cm}|}
\hline
\textbf{Model} & \textbf{Prompt} & \textbf{Model Output} & \textbf{Expected Output} & \textbf{Error Type} \\ \hline
\textbf{Base Model} 
& Translate: Sa k ap fèt? 
& What’s happening? 
& What’s going on? 
& Ambiguity in linguistic tags \\ \hline
\textbf{LDP Model} 
& Provide a formal response to: Yo, what’s good? 
& Hello, how are you? 
& Greetings, how can I assist you? 
& Misaligned reasoning \\ \hline
\textbf{Polyglot Model} 
& Explain this proverb: What goes around, comes around. 
& It’s a cycle. 
& Actions have consequences. 
& Idiomatic challenge \\ \hline
\textbf{Combined Model} 
& Rephrase: I ain’t got nothing. 
& I don’t have anything. 
& I do not have anything. 
& False negatives \\ \hline
\end{tabular}%
}
\caption{Examples of errors observed in responses.}
\label{tab:error-analysis}
\end{table}





